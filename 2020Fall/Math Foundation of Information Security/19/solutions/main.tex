\documentclass{article}

\usepackage[utf8]{inputenc}
\usepackage{ctex}
\usepackage{assignpkg}
\usepackage{amsmath}
\usepackage{amssymb}
\usepackage{threeparttable}
%\usepackage{ctable}
\studentIds{202XX80XXXXXXXX}{}
\studentNames{XXX}{}

\assignmentNumber{4}

\date{\today}

\begin{document}

% \makecover
\section*{1 一次同余方程}
该一次同余式有解的充要条件是$x$的系数9不能被模数15整除,这是显然成立的。考虑$$\frac{9}{(9,15)}x\equiv\frac{12}{(9,15)} \mod \frac{15}{(9,15)}$$
即\[3x\equiv4 \mod 5\]的解。
易知$3^{-1}\mod 5 = 2$,故其解为$x\equiv 3 \mod 5$。

而$9x\equiv 12 \mod 15$的解数为(9,15)=3个,故其解为
\[x\equiv 3,8,13 \mod 15\]
\section*{2 辗转相除法相关}
a=4864,b=3458,
$$
\begin{aligned}
4864 &= 3458*1 + 1406\\
3458 &= 1406*2 + 646\\
1406 &= 646 *2 + 114\\
646 &= 114 * 5 + 76\\
114 &= 76 + 38\\
76 &= 38*2 + 0\\
\end{aligned}
$$
故4864和3458的最大公因数是38。
$$
\begin{aligned}
38 &= 114 - 76\\
&=114 - (646-5*114)\\
&=6*114 - 646\\
&=6*(1406-646*2) - 646\\
&=6*1406 - 646*13\\
&=6*1406 - (3458-1406*2)*13\\
&=32*1406 - 3458*13\\
&=32*(4864-3458) - 3458*13\\
&=32*4864 - 45*3458
\end{aligned}
$$
故$s=32,t=45$.
将$(s,t)$减去$(\frac{3458}{38},-\frac{4864}{38})=(91,-128)$可得到$	$。
\section*{3 重复平方乘方法}
令$a=1,b=2$,$29=1+4+8+16=1+0\cdot2+1\cdot2^2+1\cdot2^3+1\cdot2^4$。
\begin{itemize}
\item[] (1)
计算$a_0\equiv a\cdot b^1 \equiv2 \mod 37$,再计算$b_1\equiv b^2\equiv 4 \mod 37$
\item[] (2)
计算$a_1\equiv a_0\cdot b_1^0 \equiv2 \mod 37$,再计算$b_2\equiv b_1^2\equiv 16 \mod 37$
\item[] (3)
计算$a_2\equiv a_1\cdot b_2^1 \equiv32 \mod 37$,再计算$b_3\equiv b_2^2\equiv 34 \mod 37$
\item[] (4)
计算$a_3\equiv a_2\cdot b_3^1 \equiv15 \mod 37$,再计算$b_4\equiv b_3^2\equiv 9 \mod 37$
\item[] (5)
计算$a_4\equiv a_3\cdot b_4^1 \equiv24 \mod 37$,再计算$b_5\equiv b_4^2\equiv 7 \mod 37$
\end{itemize}

故$2^{29}\equiv 24\mod 37$

\section*{4 中国剩余定理}
$m_1=5,m_2=6,m_3=7,m_4=11.$

令$m=5\cdot 6\cdot 7\cdot 11=2310$
\[M_1 = 6*7*11=462,M_2=5*7*11=385,\]
\[M_3 = 5*6*11=330,M_4=5*6*7=210.\]

而$M_1^{'}\equiv M_1^{-1} \equiv 2^{-1}\equiv 3\mod 5$

$M_2^{'}\equiv M_2^{-1} \equiv 1^{-1}\equiv 1\mod 6$

$M_3^{'}\equiv M_3^{-1} \equiv 1^{-1}\equiv 1\mod 7$

$M_4^{'}\equiv M_4^{-1} \equiv 1^{-1}\equiv 1\mod 11$

故$x\equiv 2*462*3+1*385*1+3*330*1+0*210*1 \equiv 1837 \mod 2310$

\section*{5 求解同余式方程组}
由于$49=7^2$,令$p=7$,考虑同余式$x^2+4x-5\equiv (x+5)(x-1) \equiv 0 \mod 7$,可验算得到其解为$x\equiv 1,2 \mod7$。
对于$x_1\equiv 1 \mod 7$,可以计算对应的$x^2+4x-5\equiv 0 \mod 49$的解$x_2$。
其中$t_1\equiv - \frac{f(x_{1})}{p}(f'(x_1)^{-1}\mod 7)\equiv - \frac{0}{7}(6^{-1}\mod 7) \equiv 0\mod 7$

故$x_2\equiv x_1+pt_1 \equiv 1 \mod 49$

对于$x'_1\equiv 2 \mod 7$,可以计算对应的$x^2+4x-5\equiv 0 \mod 49$的解$x'_2$。
其中$t'_1\equiv - \frac{f(x'_{1})}{p}(f'(x'_1)^{-1}\mod 7)\equiv - \frac{7}{7}(1^{-1}\mod 7) \equiv 6\mod 7$

故$x'_2\equiv x_1+pt_1 \equiv 2+7\cdot6 \equiv 44 \mod 49$

故$x^2+4x-5 \equiv 0 \mod 49$的解为$x\equiv 1,44\mod 49$。\\\\

$x^2+4x-5\equiv 0 \mod 27$的解。

令$p=3$,考虑同余式$x^2+4x-5 \equiv 0\mod 3$,即$x^2+x-2 \equiv 0 \mod 3$,直接验算,其解为$x\equiv 1 \mod 3$。
满足$f(x'_1) \equiv 0 \mod 3$且$f(x'_1)\equiv 0 \mod 3$的$x'_1$为$x'_1\equiv 1 \mod 3$
\begin{itemize}
\item
	$f(1)\equiv 0 \mod 9$,所以$f(x) \equiv 0 \mod 9 $存在着模9意义下模3同余于1的3个解,这三个解为$x'_2 \equiv 1,1+3,1+6 \mod 27$。
\end{itemize}
所以$f(x) \equiv 0 \mod 9$的解为$x\equiv 1,4,7 \mod 9$。\\

满足$f(x'_2) \equiv 0 \mod 9$且$f(x'_2)\equiv 0 \mod 3$的$x'_2$为$x'_2\equiv 1,4,7 \mod 9$
\begin{itemize}
\item
	$f(1)\equiv 0 \mod 27$,所以$f(x) \equiv 0 \mod 27 $存在着模27意义下模9同余于1的3个解,这三个解为$x'_2 \equiv 1,1+9,1+18 \mod 27$。
\item
	$f(4)\equiv 0 \mod 27$,所以$f(x) \equiv 0 \mod 27 $存在着模27意义下模9同余于4的3个解,这三个解为$x'_2 \equiv 4,4+9,4+18 \mod 27$。
\item
	$f(7)\equiv 18 \not\equiv 0 \mod 27$,所以$f(x) \equiv 0 \mod 27 $不存在着模27意义下模9同余于7的解。

\end{itemize}
所以$f(x) \equiv 0 \mod 27$的解为$x\equiv 1,4,10,13,19,22 \mod 27$。
\section*{6 勒让德符号}

$$
\begin{aligned}
(\frac{173}{401})&=(-1)^{\frac{173-1}{2}\frac{401-1}{2}}(\frac{401}{173})\\
&=(\frac{401}{173})\\
&=(\frac{55}{173})\\
&=(\frac{5}{173})(\frac{11}{173})\\
&=(-1)^{\frac{5-1}{2}\frac{173-1}{2}}(\frac{173}{5})(-1)^{\frac{11-1}{2}\frac{173-1}{2}}(\frac{173}{11})\\
&=(\frac{3}{5})(\frac{8}{11})\\
&=(-1)^{1\cdot2}(\frac{5}{3})(\frac{2}{11})(\frac{2}{11})(\frac{2}{11})\\
&=(\frac{2}{3})\frac{2}{11}\\
&=(-1)^{\frac{3^2-1}{8}}(-1)^{\frac{11^2-1}{8}}\\
&=1\\
(\frac{174}{401})&=(\frac{2}{401})(\frac{3}{401})(\frac{29}{401})\\
&=(-1)^{\frac{401^2-1}{8}}(-1)^{1\cdot200}(\frac{401}{3})(-1)^{14\cdot200}(\frac{401}{29})\\
&=(\frac{2}{3})(\frac{24}{29})\\
&=-1\cdot(\frac{4}{29})(\frac{2}{29})(\frac{3}{29})\\
&=-1\cdot(-1)^{\frac{29^2-1}{8}}(-1)^{1\cdot14}(\frac{2}{3})\\
&=-1
\end{aligned}
$$

\section*{7 开平方根算法}
$a=173$,对$p=401$,$p-1=400=2^4\cdot25$,即$t=4,s=25$是奇数。

\begin{itemize}
\item[] (1)
任选一个模401的平方非剩余6,即$n=6$使得$(\frac{6}{401})=-1$.
再令$b:=6^{25} \equiv 371 \mod 401$
\item[] (2)
计算 $x_3:=173^{\frac{25+1}{2}}\equiv 256 \mod 401$.
$a^{-1} = 51 \mod 401$
\item[] (3)
因为$(a^{-1}x_3^2)^{2^2}\equiv (51\cdot256^2)^4 \equiv 1 \mod 401$。
故令$j_0=0,x_2\equiv x_3b^{j_0}\equiv x_3\equiv 256 \mod 401$.
\item[] (4)
因为$(a^{-1}x_2^2)^{2}\equiv (51\cdot256^2)^2 \equiv 1 \mod 401$。
故令$j_1=0,x_1\equiv x_2b^{2\times{j_1}}\equiv x_2\equiv 256 \mod 401$.
\item[] (5)
因为$(a^{-1}x_1^2) \equiv (51\cdot256^2) \equiv 1 \mod 401$。
故令$j_2=0,x_0\equiv x_1b^{2^2\times{j_2}}\equiv x_1\equiv 256 \mod 401$.

则$x\equiv x_0 \equiv 256 \mod 401$ 满足同余式
$$x^2\equiv 173 \mod 401$$
\\\\
$(\frac{174}{401})=-1$,故
$$x^2\equiv 174 \mod 401$$
无解。
\end{itemize}

\section*{8 多项式的最大公因式}
$$
\left(
\begin{matrix}
x^5+x^3+x+1 & 1 & 0\\
x^3+x^2+x+1 & 0 & 1
\end{matrix}
\right)
\rightarrow 
\left(
\begin{matrix}
x^4+x^2+x+1 & 1 & x^2\\
x^3+x^2+x+1 & 0 & 1
\end{matrix}
\right)
$$
$$
\rightarrow 
\left(
\begin{matrix}
x^3+1 & 1 & x^2+x\\
x^3+x^2+x+1 & 0 & 1
\end{matrix}
\right)
\rightarrow 
\left(
\begin{matrix}
x^3+1 & 1 & x^2+x\\
x^3+x^2+x+1 & 0 & 1
\end{matrix}
\right)
$$
$$
\rightarrow 
\left(
\begin{matrix}
x^3+1 & 1 & x^2+x\\
x^2+x & 1 & x^2+x+1
\end{matrix}
\right)
\rightarrow 
\left(
\begin{matrix}
x+1 & x & x^3+x^2+x+1\\
x^2+x & 1 & x^2+x+1
\end{matrix}
\right)$$
$$
\rightarrow
\left(
\begin{matrix}
x+1 & x & x^3+x^2+x+1\\
0 & 0 & x^4+x^3+1
\end{matrix}
\right)
$$
故$x+1$是$x^5+x^3+x+1$和$x^3+x^2+x+1$的最大公因式。
且$x+1 = x(x^5+x^3+x+1) + (x^3+x^2+x+1)(x^3+x^2+x+1)$,
即$s(x) = x,	t(x) = (x^3+x^2+x+1)$
\clearpage
\section*{9 8元域上的加法表和乘法表}
% 加法表:

% \begin{tabular}{ccccccccc}
% 			& 0 		& 1 		& $x$ 		& $x+1$ 	& $x^2$ 	& $x^2+1$ 	& $x^2+x$ 	& $x^2+x+1$ \\
% 0 			& 0			& 1 		& $x$ 		& $x+1$ 	& $x^2$ 	& $x^2+1$ 	& $x^2+x$ 	& $x^2+x+1$ \\ 
% 1			& 1			& 0			& $x+1$ 	& $x$ 		& $x^2+1$	& $x^2$		& $x^2+x+1$	& $x^2+x$	\\
% $x$ 		& $x$		& $x+1$		& 0 		& 1 		& $x^2+x$	& $x^2+x+1$ & $x^2$		& $x^2+1$	\\
% $x+1$ 		& $x+1$		& $x$		& 1 		& 0 		& $x^2+x+1$ & $x^2+x$	& $x^2+1$	& $x^2$		\\
% $x^2$		& $x^2$		& $x^2+1$	& $x^2+x$	& $x^2+x+1$ & 0 		& 1			& $x$		& $x+1$		\\
% $x^2+1$		& $x^2+1$	& $x^2$		& $x^2+x+1$	& $x^2+x$	& 1			& 0			& $x+1$		& $x$		\\
% $x^2+x$ 	& $x^2+x$	& $x^2+x+1$	& $x^2$ 	& $x^2+1$	& $x$		& $x+1$		& 0			& 1			\\
% $x^2+x+1$ 	& $x^2+x+1$	& $x^2+x$	& $x^2+x+1$	& $x^2$		& $x+1$		& $x$		& 1			& 0			
% \end{tabular}\\\\

% 乘法表

% \begin{tabular}{ccccccccc}
% 			& 0 		& 1 		& $x$ 		& $x+1$ 	& $x^2$ 	& $x^2+1$ 	& $x^2+x$ 	& $x^2+x+1$ \\
% 0			& 0			& 0			& 0			& 0			& 0			& 0			& 0			& 0			\\
% 1 			& 0			& 1 		& $x$ 		& $x+1$ 	& $x^2$ 	& $x^2+1$ 	& $x^2+x$ 	& $x^2+x+1$ \\ 
% $x$ 		& 0 		& $x$		& $x^2$		& $x^2+x$	& $x+1$		& $1$		& $x^2+x+1$	& $x^2+1$	\\
% $x+1$ 		& 0 		& $x+1$		& $x^2+x$	& $x^2+1$	& $x^2+x+1$	& $x^2$		& 1			& $x$		\\
% $x^2$		& 0 		& $x^2$		& $x+1$		& $x^2+x+1$	& $x^2+x$	& $x$		& $x^2+1$	& 1			\\
% $x^2+1$		& 0 		& $x^2+1$	& $1$		& $x^2$		& $x$		& $x^2+x+1$	& $x+1$		& $x^2+x$	\\
% $x^2+x$ 	& 0 		& $x^2+x$	& $x^2+x+1$	& 1			& $x^2+1$	& $x+1$		& $x$		& $x^2$		\\
% $x^2+x+1$ 	& 0 		& $x^2+x+1$	& $x^2+1$	& $x$		& 1			& $x^2+x$	& $x^2$		& $x+1$		
% \end{tabular}
\begin{center}
\begin{table*}[h]
\resizebox{\textwidth}{!}{ %
\begin{threeparttable}[b]
\caption{加法表}
\label{Tab:bondlength}
\begin{tabular}{l|llllllll}
			& 0 		& 1 		& $x$ 		& $x+1$ 	& $x^2$ 	& $x^2+1$ 	& $x^2+x$ 	& $x^2+x+1$ \\
\hline
0 			& 0			& 1 		& $x$ 		& $x+1$ 	& $x^2$ 	& $x^2+1$ 	& $x^2+x$ 	& $x^2+x+1$ \\ 
1			& 1			& 0			& $x+1$ 	& $x$ 		& $x^2+1$	& $x^2$		& $x^2+x+1$	& $x^2+x$	\\
$x$ 		& $x$		& $x+1$		& 0 		& 1 		& $x^2+x$	& $x^2+x+1$ & $x^2$		& $x^2+1$	\\
$x+1$ 		& $x+1$		& $x$		& 1 		& 0 		& $x^2+x+1$ & $x^2+x$	& $x^2+1$	& $x^2$		\\
$x^2$		& $x^2$		& $x^2+1$	& $x^2+x$	& $x^2+x+1$ & 0 		& 1			& $x$		& $x+1$		\\
$x^2+1$		& $x^2+1$	& $x^2$		& $x^2+x+1$	& $x^2+x$	& 1			& 0			& $x+1$		& $x$		\\
$x^2+x$ 	& $x^2+x$	& $x^2+x+1$	& $x^2$ 	& $x^2+1$	& $x$		& $x+1$		& 0			& 1			\\
$x^2+x+1$ 	& $x^2+x+1$	& $x^2+x$	& $x^2+1$	& $x^2$		& $x+1$		& $x$		& 1			& 0			
\end{tabular}
\end{threeparttable}}%
\end{table*}
\end{center}


\begin{center}
\begin{table*}[h]
\resizebox{\textwidth}{!}{ %
\begin{threeparttable}[b]
\caption{乘法表}
\label{Tab:bondlength}
\begin{tabular}{l|llllllll}
			& 0 		& 1 		& $x$ 		& $x+1$ 	& $x^2$ 	& $x^2+1$ 	& $x^2+x$ 	& $x^2+x+1$ \\
\hline
0			& 0			& 0			& 0			& 0			& 0			& 0			& 0			& 0			\\
1 			& 0			& 1 		& $x$ 		& $x+1$ 	& $x^2$ 	& $x^2+1$ 	& $x^2+x$ 	& $x^2+x+1$ \\ 
$x$ 		& 0 		& $x$		& $x^2$		& $x^2+x$	& $x+1$		& $1$		& $x^2+x+1$	& $x^2+1$	\\
$x+1$ 		& 0 		& $x+1$		& $x^2+x$	& $x^2+1$	& $x^2+x+1$	& $x^2$		& 1			& $x$		\\
$x^2$		& 0 		& $x^2$		& $x+1$		& $x^2+x+1$	& $x^2+x$	& $x$		& $x^2+1$	& 1			\\
$x^2+1$		& 0 		& $x^2+1$	& $1$		& $x^2$		& $x$		& $x^2+x+1$	& $x+1$		& $x^2+x$	\\
$x^2+x$ 	& 0 		& $x^2+x$	& $x^2+x+1$	& 1			& $x^2+1$	& $x+1$		& $x$		& $x^2$		\\
$x^2+x+1$ 	& 0 		& $x^2+x+1$	& $x^2+1$	& $x$		& 1			& $x^2+x$	& $x^2$		& $x+1$		
\end{tabular}
\end{threeparttable}}%
\end{table*}
\end{center}
\end{document}



