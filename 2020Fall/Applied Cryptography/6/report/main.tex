\documentclass{article}

\usepackage[utf8]{inputenc}
\usepackage{ctex}
\usepackage{assignpkg}
\usepackage{amsmath}
\usepackage{amssymb}
\studentIds{202XX80XXXXXXXX}{}
\studentNames{XXX}{}

\assignmentNumber{6}

\date{\today}

\begin{document}

\makecover
\section*{1 DSA签名}
若$s=0$,则$k^{-1}(SHA(M)+xr) = 0 \mod q$。而$k^{-1}$必不等于$0 \mod q$,故$SHA(M)+xr = 0 \mod q$。

而由于签名中包含$M$,故可以计算出$SHA(M)$,且$r$也已知,则可计算私钥$x = -SHA(M)r^{-1} \mod q$。即$s=0$时,私钥是可以被计算出来的。所以应该避免这种情况。

\section*{2 ElGamal签名}
\begin{itemize}
	\item[] (1)
		本题中,字母$\beta$表示的应当是公钥,$v$表示的是随机数$k$。

		要证明$(r,s)$数对是消息$m=su \mod (p-1)$的一个有效签名,只需验证$\alpha^m=\beta^rr^s$即可。
		$$
		\begin{aligned}
		\alpha^m&=\alpha^{su} 
		&=(\alpha^u)^s
		&=(r\beta^{-v})^s
		&=r^s \beta^{-vs}
		&=r^s \beta^{-v(-rv^{-1})}
		&=\beta^{r}r^s
		\end{aligned}
		$$
		验证成立
	\item[] (2)
		若采用对消息的散列函数进行签名,则需要验证签名的等式应当为$\alpha^{h(m)}=\beta^rr^s$,那么伪造消息就要找到一个消息,使得其散列值$h(m)=su$。总所周知,由于散列函数的单向性,对于某一个固定的散列输出,要找到其可能对应的某个输入是困难的,所以攻击者很难找到一个$m$,使得$h(m)=su$,故能抵御存在伪造攻击。
\end{itemize}

\section*{3 Shamir秘密分享}
$$
\begin{aligned}
f(x)&=y_1\frac{(x-x_2)(x-x_3)}{(x_1-x_2)(x_1-x_3)} + y_2\frac{(x-x_1)(x-x_3)}{(x_2-x_1)(x_2-x_3)} + y_3\frac{(x-x_1)(x-x_2)}{(x_3-x_1)(x_3-x_2)}\\
&=8\frac{(x-3)(x-5)}{8} + 10\frac{(x-1)(x-5)}{-4} + 11\frac{(x-1)(x-3)}{8}\\
&=(x-3)(x-5)+6(x-1)(x-5)+12(x-1)(x-3)\\
&=2x^2+10x+13
\end{aligned}
$$
故秘密为13。
\section*{4 公平猜拳游戏}
设猜拳的两个人分别为$Alice$和$Bob$,由于出拳的方式有三种,故每次出拳都包含大约两比特信息。
规定,$Alice$和$Bob$的通信,其要传递的消息$m$包含两比特,出拳为“锤子”时,$m=00$;出拳为“剪刀”时,$m=01$;出拳为“步”时,$m=10$;现规定游戏流程如下:
\begin{itemize}
\item[] (a)
	$Alice$随机选择$r_a$,并选择自己的出拳方式$m_a$,利用$Hash$函数$h$计算:$H_a=h(r_a,m_a)$,然后把$H_a$发送给$Bob$;
\item[] (b)
	$Bob$随机选择$r_b$,并选择自己的出拳方式$m_b$,利用$Hash$函数$h$计算:$H_b=h(r_b,m_b)$,然后把$H_b$发送给$Alice$;
\item[] (c)
	$Alice$将$r_a,m_a$发送给$Bob$;
\item[] (d)
	$Bob$将$r_b,m_b$发送给$Alice$;
\item[] (e)
	$Alice$和$Bob$验证对方发送的随机数、消息和哈希值是否符合$H=h(r,m)$,并确定这次猜拳的结果。
\end{itemize}
说明:步骤$a,b$为出拳方式,没有顺序;步骤$c,d$也没有顺序,但是步骤$c,d$一定要在步骤$a,b$完成之后才可以进行。
\end{document}






