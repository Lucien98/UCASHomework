\documentclass{article}

\usepackage[utf8]{inputenc}
\usepackage{ctex}
\usepackage{assignpkg}
\usepackage{amsmath}
\usepackage{amssymb}
\studentIds{202XX80XXXXXXXX}{}
\studentNames{XXX}{}

\assignmentNumber{2}

\date{\today}

\begin{document}

\makecover

\section*{信息熵的计算}
\begin{itemize}
    \item [(1)]
\[H(M) = \sum_{i=1}^{3}p(m_i)I(m_i) = -(\frac{1}{3}\log_2{\frac{1}{3}} 
+ \frac{8}{15}\log_2{\frac{8}{15}} + \frac{2}{15}\log_2{\frac{2}{15}}) = 1.40\]
\item [(2)]
\[H(K)=\sum_{i=1}^{3}p(k_i)I(k_i)=-(\frac{1}{4}\log_2{\frac{1}{4}}+\frac{1}{2}\log_2{\frac{1}{2}}+
\frac{1}{4} \log_2{\frac{1}{4}})=\frac{3}{2}=1.5\]
\item [(3)]
\[p(c=1) = p(m=a)p(k=k_3)+p(m=c)p(k=k_2) = \frac{1}{3}\frac{1}{4}+\frac{2}{15}\frac{1}{4}=\frac{7}{60}\]
\[p(c=2) = p(m=a)p(k=k_1)+p(m=b)p(k=k_3) = \frac{1}{3}\frac{1}{2} + \frac{8}{15}\frac{1}{4}=\frac{3}{10}\]
\[p(c=3) = p(m=a)p(k=k_2) +p(m=b)p(k=k_1) + p(m=c)p(k=k_3) = \frac{1}{3}\frac{1}{4} + \frac{8}{15}\frac{1}{2} + \frac{2}{15}\frac{1}{4} = \frac{23}{60}\]
\[p(c=4) = p(m=b)p(k=k_2) + p(m=c)p(k=k_1) = \frac{8}{15}\frac{1}{4} + \frac{2}{15}\frac{1}{2} = \frac{1}{5} \]
\[H(C)=\sum_{i=0}^{4}p(c_i)I(c_i) = -(\frac{7}{60}\log_2{\frac{7}{60}} + \frac{3}{10}\log_2\frac{3}{10} + \frac{23}{60}\log_2\frac{23}{60} + \frac{1}{5}\log_2{\frac{1}{5}})=1.877\]

\item [(4)]
联合概率、条件概率表如下:

\begin{tabular}{ccccc}
\hline
$p(m_i,c_j)$ & 1 & 2 & 3 & 4 \\
\hline
a & $\frac{1}{12}$ & $\frac{1}{6}$ & $\frac{1}{12}$ & 0 \\
b & 0 & $\frac{2}{15}$ & $\frac{4}{15}$ & $\frac{2}{15}$ \\
c & $\frac{1}{30}$ & 0 & $\frac{1}{30}$ & $\frac{1}{15}$ \\
\hline
\end{tabular}
\begin{tabular}{ccccc}
\hline
$p(m_i|c_j)$&1&2&3&4\\
\hline
a & $\frac{5}{7}$ & $\frac{5}{9}$ & $\frac{5}{23}$ & 0 \\
b & 0 & $\frac{4}{9}$ & $\frac{16}{23}$ & $\frac{2}{3}$ \\
c & $\frac{2}{7}$ & 0 & $\frac{2}{23}$ & $\frac{1}{3}$ \\
\hline
\end{tabular}

\[ H(M|C) = \sum_{i,j} p(m_i,c_j)I(m_i|c_i)=-(\frac{1}{12}\log_2\frac{5}{7}+\frac{1}{6}\log_2\frac{5}{9}+\frac{1}{12}\log_2\frac{5}{23}+\]
\[\frac{2}{15}\log_2\frac{4}{9}+\frac{4}{15}\log_2\frac{16}{23}+\frac{2}{15}\log_2\frac{2}{3}+\frac{1}{30}\log_2\frac{2}{7}+\frac{1}{30}\log_2\frac{2}{23}+\frac{1}{15}\log_2\frac{1}{3})=1.022\]

\item [(5)]
联合概率、条件概率表如下:

\begin{tabular}{ccccc}
\hline
$p(k_i,c_j)$ & 1 & 2 & 3 & 4 \\
\hline
$k_1$ & 0 & $\frac{1}{6}$ & $\frac{4}{15}$ & $\frac{1}{15}$ \\
$k_2$ & $\frac{1}{30}$ & 0 & $\frac{1}{12}$ & $\frac{2}{15}$ \\
$k_3$ & $\frac{1}{12}$ & $\frac{2}{15}$ & $\frac{1}{30}$ & 0 \\
\hline
\end{tabular}
\begin{tabular}{ccccc}
\hline
$p(k_i|c_j)$&1&2&3&4\\
\hline
$k_1$ & 0 & $\frac{5}{9}$ & $\frac{16}{23}$ & $\frac{1}{3}$ \\
$k_2$ & $\frac{2}{7}$ & 0 & $\frac{5}{23}$ & $\frac{2}{3}$ \\
$k_3$ & $\frac{5}{7}$ & $\frac{4}{9}$ & $\frac{2}{23}$ & 0 \\
\hline
\end{tabular}

\[ H(K|C) = \sum_{i,j} p(k_i,c_j)I(k_i|c_i)=1.022\]
\end{itemize}


\section*{第二题}

\begin{itemize}
    \item [(a)]
    \[\because H(C,P,K) = H(P,K) + H(C | P,K)\]
    其中$H(C|P,K)$表示已知明文和密钥之后,密文还保留的信息量,此时密文还有的信息量为零,故$H(C|P,K)=0$
    \[\therefore H(C,P,K) = H(P,K)\]
    又因为密码系统中明文和密文的分布是独立的,所以$H(P,K)=H(P)+H(K)$
    故\[H(P,K)=H(C,P,K)=H(P)+H(K)\]
    \item [(b)]
    \begin{itemize}
        \item [(1)] 
            $\because H(C,P)=H(P|C) + H(C)$
            ,又$\because$在完善保密系统中,密文不会透露出明文的任何信息,即明文和密文互相独立,则$H(P|C)=H(P)$。
            故$H(C,P)=H(P) + H(C)$
        \item [(2)]
             \[H(C)=H(C,P)-H(P)=H(C,P,K)-H(K|C,P)-H(P) \]
            \[ =H(P)+H(K)-H(K|C,P)-H(P)=H(K)-H(K|C,P)\]
               
    \end{itemize}

    \item [(c)]
    因为在完善保密系统中,有(b)中结论成立,且明密文对有唯一密钥,故当明密文确定时,密钥也唯一确定,故
    $H(K|C,P) = 0$,则有$H(C)=H(K)-H(K|C,P)=H(K)$

\section*{第三题}
    设密钥空间为KEY,\[S_1(x)=x+k_1,\ k_1\sim U(KEY)\]
    \[S_2(x)=x+k_2,\ k_2 \sim P_k\]
    其中$U(KEY)$表示在密钥空间$KEY$上的均匀分布,$P_k$为$k_2$的概率分布。
    则\[S_1*S_2(x)=x+k_1+k_2\]
    令$k=k_1+k_2$
    任取$K\in KEY$
    \[p(k=K)=\sum_{i=1}^{|k_2|} p(k_2=K_2)p(k_1=K-K_2) \]
    \[= p(k_1=K-K_2)\sum_{i=1}^{|k_2|} p(k_2=K_2)=p(k_1=K-K_2)=\frac{1}{|KEY|}\]
    故可知$k=k_1+k_2$也是服从均匀分布,即$k \sim U(KEY)$
    故$S_1*S_2=S_1$。
    此处定义的相等为密钥在密钥空间的分布是一致的。
\end{itemize}

\[f(-1)=\frac{1}{f(-1+2)}=\frac{1}{f(1)}=\frac{1}{\frac{1}{f(1+2)}}=f(1+2)=f(3)=\frac{1}{f(3+2)}=\frac{1}{f(5)}=\frac{1}{\frac{1}{f(5+2)}}=f(7)=2\]

\end{document}
