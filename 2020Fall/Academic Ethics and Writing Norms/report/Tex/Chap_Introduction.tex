\chapter{引言部分的分析}\label{chap:introduction}

在Introduction部分,作者完成了五部分内容的说明,分别对应文章中的五段。

第一段,作者对图像分类这一领域简单的做了背景和现状的介绍,这一段内容相信不只是小同行,大同行都能轻松读懂。

第二段作者主要说明了如果在大规模数据集下使用前人方法会遇到的问题和挑战——任务的复杂性会非常高。

第三段作者根据上一段的问题和挑战提出了解决思路,Gpu和2D卷积相结合可以训练大型的CNN,且带标签数据集可以使模型不发生严重的过拟合。

第四段作者通过和前人方法的效果对比说明论文的贡献,在说明使用方法的同时,说明了文章的具体结构。

最后,在引言第五段,作者提出了可能存在的改进方法——只需要等更快的GPU和更大的数据集就可以得到更好的结果。

这五个部分内容同刘老师上课讲的完全一致,拥有了目的、核心任务、思路、贡献和文章结构5个要素。通过阅读引言,就能明白了作者想要解决什么样的问题,且该文作者只着重讲了一个问题的解决,也就是只有一个卖点——在大规模图片数据集上进行对象识别得到了比其他方法更好的结果,而这一点也是刘老师上课时强调的。

值得一提的是,该文提出问题的方式是“补”,也就是提出了Gap,在前人工作的基础上进行了改良,得到了比前人更好的结果。

纵观整个引言部分,作者精雕细琢,没有多写一句废话,没有介绍常识,也没有提及具体细节,对他人的工作也非常客观的做出了评价。

