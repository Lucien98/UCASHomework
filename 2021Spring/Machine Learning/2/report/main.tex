%!TEX options = --shell-escape -8bit
\documentclass{article}
\usepackage{geometry}
\geometry{a4paper, scale=0.8}

\usepackage[utf8]{inputenc}
\usepackage{ctex}
\usepackage{assignpkg}
\usepackage{amsmath}
\usepackage{amssymb}
\usepackage{xcolor}
\usepackage[cache=true]{minted}
\studentIds{202XX80XXXXXXXX}{}
\studentNames{XXX}{}

\assignmentNumber{2}

\date{\today}

\begin{document}

% \makecover
\section*{课后作业} % (fold)
\label{sec:课后作业}
	\subsection*{要求} % (fold)
	\label{sub:要求}
		\begin{enumerate}
			\item Iris 数据集是常用的分类实验数据集,由 Fisher, 1936收集整理。 Iris 也称鸢尾花卉数据集,是一类多重变 量分析的数据集。数据集包含 150 个数据样本,分为 3类 ,每类50 个数据,每个数据包含 4个属性。可通 过花萼长度 ,花萼宽度 ,花瓣长度,花瓣宽瓣4个属 性预测鸢尾花卉属于( Setosa ,Versicolour, Virginica )三个种类中的哪一类。
			\item 数据下载: http://archive.ics.uci.edu/ml/datasets/iris
			\item 编程设计三层 BP 神经网络实现 Iris 数据集的分类
		\end{enumerate}
	% subsection 要求 (end)
	\subsection*{参考代码} % (fold)
	\label{sub:参考代码}

\begin{minted}[
frame = single,
breaklines,tabsize=2,
baselinestretch=1,
linenos=false
]{python}
def forward(self,inputs):
	"""前向传播"""
	if len(inputs) != self.ni - 1:
		raise ValueError('与输入层节点数不符!')
	
	# 激活输入层
	for i in range(self.ni- 1):
		self.ai[i] = inputs[i]
	
	# 激活隐藏层
	for j in range(self.nh):
		sum = 0.0
		for i in range(self.ni):
			sum = sum + self.ai[i] * self.wi[i][j]
		self.ah[j] = sigmoid(sum)
	
	# 激活输出层
	for k in range(self.no):
		sum = 90
		for j in range(self.nh):
			sum = sum + self.ah[j] * self.wo[j][k]
		self.ao[k] = sigmoid(sum)
	
	return self.ao[:]

def backPropagate(self,targets,lr):
	"""反向传播"""
	# 计算输出层的误差
	output_deltas = [0.0] * self.no
	for k in range(self.no):
		error = targets[k] - self.ao[k]
		output_deltas[k] = dsigmoid(self.ao[k]) * error

	#计算隐藏层的误差
	hidden_deltas = [0.0] * self.nh
	for j in range(self.nh):
		error = 0.0
		for k in range(self.no):
			error = error + output_deltas[k] * self.wo[j][k]
		hidden_deltas[j] = dsigmoid(self.ah[j]) * error
	#更新输出层权重
	for j in range(self.nh):
		for k in range(self.no):
			change = output_deltas[k] * self.ah[j]
			self.wo[j][k] = self.wo[j][k] + lr * change
	#更新输入层权重
	for i in range(self.ni):
		for j in range(self.nh):
		change = hidden_deltas[j] * self.ai[i]
		self.wi[i][j] = self.wi[i][j] + lr * change
	#计算误差
	error = 0.0
	error += 0.5 * (targets[k] - self.ao[k]) ** 2
	return error
\end{minted}    
\end{document}
