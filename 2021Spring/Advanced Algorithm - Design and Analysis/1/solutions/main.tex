\documentclass{article}
\usepackage{geometry}
\geometry{a4paper,scale=0.8}

\usepackage[utf8]{inputenc}
\usepackage{ctex}
\usepackage{assignpkg}
\usepackage{amsmath}
\usepackage{amssymb}
\studentIds{202XX80XXXXXXXX}{}
\studentNames{XXX}{}

\assignmentNumber{1}

\date{\today}

\begin{document}

\makecover

\section*{1}
先证$(\frac{n}{m})^m\leq \binom{n}{m}$。当$0\leq i < m\leq n$时,有$\frac{n-i}{m-i}\geq\frac{n}{m}$
\begin{equation*}
\begin{split}
\binom{n}{m}&=\prod_{i=0}^{m-1}\frac{n-i}{m-i}\\
&\leq\prod_{i=0}^{m-1}\frac{n}{m}\\
&=(\frac{n}{m})^m
\end{split}
\end{equation*}

得证。
\\\\

再证$\binom{n}{m}\leq (\frac{ne}{m})^m$。采用数学归纳法。

当$m=1\leq n$时,$\binom{n}{m}=\binom{n}{1}=n \leq (\frac{ne}{1})^1=ne$,该不等式成立。

当$m=2\leq n$时,$\binom{n}{m}=\binom{n}{2}=\frac{n(n-1)}{2} \leq \frac{n^2e^2}{4} = (\frac{ne}{m})^m$,该不等式成立。

不难验证,归纳假设只需验证当不等式在$(m-1,n-1)$成立时,其在$(m,n)$也成立,就可证明原不等式在对任意的$(m,n)$都成立。

假设当$2\leq m \leq n$时,不等式在$(m-1,n-1)$成立,即$\binom{n-1}{m-1}\leq (\frac{(n-1)e}{m-1})^{m-1}$。

首先可以证明:$\frac{(\frac{ne}{m})^m}{(\frac{(n-1)e}{m-1})^{m-1}}\geq \frac{n}{m}$。证明如下:

\begin{equation*}
\begin{split}
\frac{(\frac{ne}{m})^m}{(\frac{(n-1)e}{m-1})^{m-1}}&=\frac{n}{m}e(\frac{n(m-1)}{m(n-1)})^{m-1}\\
&\geq \frac{n}{m}(1+\frac{1}{m-1})^{m-1}(\frac{n(m-1)}{m(n-1)})^{m-1}\\
&=\frac{n}{m}(\frac{m}{m-1})^{m-1}(\frac{n(m-1)}{m(n-1)})^{m-1}\\
&=\frac{n}{m}(\frac{n}{n-1})^{m-1}\\
&\geq \frac{n}{m}
\end{split}
\end{equation*}

则在$(m,n)$处,
\begin{equation*}
\begin{split}
\binom{n}{m}&=\binom{n-1}{m-1}\frac{n}{m}\\
&\leq (\frac{(n-1)e}{m-1})^{m-1}\frac{(\frac{ne}{m})^m}{(\frac{(n-1)e}{m-1})^{m-1}}\\
&= (\frac{ne}{m})^m
\end{split}
\end{equation*}

得证。

\clearpage
\section*{2}
第$i$次才第一次出现正面的概率为$P(T=i) = (1-p)^{i-1}p$
% \begin{equation*}
% \begin{split}
% \mathbb{E}(T)&=\sum_{n=1}^{+\infty}iP(T=i)\\
% &=\sum_{n=1}^{+\infty} i(1-p)^{i-1}p
% \end{split}
% \end{equation*}

已知级数$$\sum_{n=1}^{+\infty} nx^{n-1}=(\sum_{1}^{+\infty}x^n)^{'}=(\frac{1}{1-x})^{'}=\frac{1}{(1-x)^2}$$

则
\begin{equation*}
\begin{split}
\mathbb{E}(T)&=\sum_{n=1}^{+\infty}iP(T=i)\\
&=\sum_{n=1}^{+\infty} i(1-p)^{i-1}p\\
&=p(\sum_{n=1}^{+\infty} i(1-p)^{i-1})\\
&=p(\frac{1}{(1-(1-p))^2})\\
&=\frac{1}{p}
\end{split}
\end{equation*}

已知级数
$$\sum_{n=1}^{+\infty}n^2x^{n-1}=(\sum_{n-1}^{+\infty}nx^n)^{'}=(x(\sum_{n=1}^{+\infty}nx^{n-1}))^{'}=(\frac{x}{(x-1)^2})^{'}=\frac{x+1}{(1-x)^3}$$

则
\begin{equation*}
\begin{split}
\mathbb{E}(T^2)&=\sum_{n=1}^{+\infty}i^2P(T=i)\\
&=\sum_{n=1}^{+\infty} i^2(1-p)^{i-1}p\\
&=p(\sum_{n=1}^{+\infty} i^2(1-p)^{i-1})\\
&=p(\frac{1-p+1}{(1-(1-p))^3})\\
&=\frac{2-p}{p^2}
\end{split}
\end{equation*}

故$\mathbb{D}T=\mathbb{E}(T^2)-(\mathbb{E}T)^2=\frac{1-p}{p^2}$

\clearpage
\section*{3}
证明:

取$\lambda>0$。

\begin{align*}
Pr(X\leq (1-\delta)\mu)&=Pr(e^{\lambda X}\le e^{\lambda (1-\delta)\mu}) \\
& \le \frac{\mathbb{E}(e^{\lambda X})}{e^{\lambda (1-\delta)\mu}} &\text{Markov不等式}\\
& = \frac{\mathbb{E}(\prod_{i=1}^ne^{\lambda Y_i})}{e^{\lambda (1-\delta)\mu}}\\
& = \frac{\prod_{i=1}^n\mathbb{E}(\lambda Y_i)}{e^{\lambda (1-\delta)\mu}}\\
& = \frac{\prod_{i=1}^n(p_ie^{\lambda}+1-p_i)}{e^{\lambda (1-\delta)\mu}}\\
& \le \frac{\prod_{i=1}^ne^{p_i(e^\lambda-1)}}{e^{\lambda (1-\delta)\mu}} & e^x+1\ge x\\
& = \frac{e^{(e^{\lambda}-1)(\sum_{i-1}^np_i)}}{e^{\lambda (1-\delta)\mu}}\\
& = \frac{e^{(e^{\lambda}-1)\mu}}{e^{\lambda (1-\delta)\mu}}\\
& = (\frac{e^{e^{\lambda}-1}}{e^{\lambda (1-\delta)}})^{\mu}\\
\end{align*}

令$\lambda=\ln (1-\delta)$,则$(\frac{e^{e^{\lambda}-1}}{e^{\lambda (1-\delta)}})^{\mu}=(\frac{e^{-\delta}}{e^{(1-\delta)\ln (1-\delta)}})^{\mu}=(\frac{e^{-\delta}}{(1-\delta)^{1-\delta}})^{\mu}
$

我们将$\frac{e^{-\delta}}{(1-\delta)^{1-\delta}}$取对数,则
\begin{align*}
\ln \frac{e^{-\delta}}{(1-\delta)^{1-\delta}}&=-\delta-(1-\delta)\ln (1-\delta)\\
&=-\delta - (1-\delta)(\sum_{n=1}^{+\infty}-\frac{-\delta^n}{n})\\
&=-\delta - (-\delta+\frac{\delta^2}{2}+\sum_{n=3}^{+\infty}\frac{1}{n(n-1)}\delta^n)\\
&=-\frac{\delta^2}{2}-(\sum_{n=3}^{+\infty}\frac{1}{n(n-1)}\delta^n)\\
& \le -\frac{\delta^2}{2}
\end{align*}

故$(\frac{e^{-\delta}}{(1-\delta)^{1-\delta}})^{\mu}\le e^{-\frac{\delta^2}{2}\mu}$

\clearpage

\section*{4}
定义随机变量$X_i,i=1,\cdots,|V|,$
\begin{equation*}
X_i=
\begin{cases}
0, &\text{顶点$u_i \in A$}\\
1, &\text{顶点$u_i \notin A$}
\end{cases}
\end{equation*}
则有Pr($X_i=0$)=Pr($X_i=1$)=$\frac{1}{2}$。

对于边集$E$中的任意一条边$(u_i,u_j)$,有
$$\Pr((u_i,u_j)\in E(A,B))=\Pr(X_i\neq X_j)=\frac{1}{2}$$

对于任意的$(u_i,u_j)\in E$,定义随机变量$Y_{i,j}$如下,
\begin{equation*}
Y_{i,j}=
\begin{cases}
0, &(u_i,u_j) \notin E(A,B)\\
1, &(u_i,u_j) \in E(A,B)
\end{cases}
\end{equation*}

则$\mathbb{E}(Y_{i,j})=\frac{1}{2},\mathbb{D}(Y_i,j)=\frac{1}{4}$
\begin{align*}
\mathbb{D}(|E(A,B)|)&=\mathbb{D}[\sum_{(u_i,u_j)\in E}Y_{i,j}]\\
&=\sum_{(u_i,u_j)\in E} \mathbb{D}(Y_{i,j})\\
&=\frac{|E|}{4}
\end{align*}
\section*{5}
记图$G$中的顶点个数为$n$,并将其分别编号为1,2,……,n。假设图中恰好存在一个完美匹配。

设计的算法描述如下:
\begin{itemize}
\item[1]
根据图中的边颜色等信息构造一个$n$阶方阵$A$,其每一项定义为:
\begin{equation*}
A(i,j) = 
\begin{cases}
y, & e = (i,j) \in E\ and\ c(e)=red\\
1, & e = (i,j) \in E\ and\ c(e)=blue\\
0, & otherwise
\end{cases}
\end{equation*}

\item[2]
随机选择n+1个数,$y_0,y_1,\cdots,y_n$,分别以这$n+1$个数代y,计算$det(A)$,得到$n+1$个对应的值$p(y_0),p(y_2),\cdots,p(y_n)$。用这$n+1$个点$\{(y_0,p(y_0)),(y_1,p(y_1)),\cdots,(y_n,p(y_n))\}$进行拉格朗日插值。得到一个多项式函数$p(y)$。

\item[3]
若拟合出来的多项式包含$\pm y^k$项,则输出$Yes$。

若拟合出来的多项式不包含$\pm y^k$项,则输出$No$。
\end{itemize}

分析:这样拟合出来的多项式是唯一的吗?

回答:这样拟合出来的多项式是唯一的,因为根据$A(i,j)$的定义,能计算出$A$的行列式是关于$y$的多项式函数,这个多项式函数是唯一的。

分析:若不存在$\pm y^k$项,那么是否可能是因为存在多个红蓝匹配,它们由于符号的原因互相抵消了呢?

回答:这种情况也是可能存在的,而且这个问题可能无法通过多次运行上述算法解决。因为多项式函数是唯一的,选取其他点进行拟合也只能拟合出一样的结果。问题的解决办法暂无。但是如果假设图中只存在一个红蓝匹配,那么这个算法将是始终有效的。

分析:为什么需要选取$n+1$个点进行拟合?

回答:因为$det(A)$最多是关于$y$的$n$次函数,而拟合$n$次函数需要$n+1$个点确定该函数的$n+1$个系数。
\end{document}
