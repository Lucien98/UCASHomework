\documentclass{article}
\usepackage{geometry}
\geometry{a4paper, scale=0.8}

\usepackage[utf8]{inputenc}
\usepackage{ctex}
\usepackage{assignpkg}
\usepackage{amsmath}
\usepackage{amssymb}
\studentIds{202XX80XXXXXXXX}{}
\studentNames{XXX}{}

\assignmentNumber{5}

\date{\today}

\begin{document}

\makecover

\subsubsection*{第一题}
\begin{itemize}
\item[a.]
$p(x_1, x_2, x_3, x_4, x_5, x_6) = p(x_1)p(x_2)p(x_3 | x_1,x_2)p(x_5 | x_2)p(x_4 | x_3)p(x_6 | x_3, x_5) $

\item[b.]
随机变量$x_1$和$x_6$之间存在路径$(x_1 , x_3, x_6)$,根据D-Seperation定理,顶点$x_3$为head-to-tail节点,需要出现在已观测集合中,$x_1$和$x_6$才独立。也就是说,$x_1$和$x_6$不独立,但是在给定$x_3$的条件下,两者独立。

\item[c.]
随机变量$x_1$和$x_5$之间存在路径$(x_1, x_3, x_2, x_5)$和$(x_1, x_3, x_6, x_5)$。由于$x_3$和$x_6$为head-to-head节点,$x_2$为tail-to-tail节点。故$x_1$和$x_5$不独立。若要$x_1$和$x_5$独立,则须满足如下条件之一:
\begin{itemize}
\item $x_3$或其后代节点不在观测的集合C中
\item $x_6$或其后代节点不在观测的集合C中
\item $x_2$在已观测的集合C中
\end{itemize}
\item[d.]
根据c.中的分析知,给定$x_3$的后代节点$x_4$的条件下,$x_1$和$x_5$独立。

\end{itemize}

\subsubsection*{第二题}
\begin{itemize}
\item[1] 最大团为AB和BC
\item[2] 联合概率分布函数

记A,B,C中的随机变量为$x_A, x_B, x_C$。

$p(A, B, C) = \frac{1}{Z}\psi_{AB}(x_A, x_B)\psi_{BC}(x_B, x_C)$
\item[3] 证明

\begin{equation*}
\begin{split}
p(A,C|B) &= \frac{p(A, B, C)}{p(B)} = \frac{p(A, B, C)}{\sum_{x'_{A}}\sum_{x'_{C}}\frac{1}{Z}\psi_{AB}(x'_{A}, x_B)\psi_{BC}(x_B, x'_{C})}\\
&=\frac{\frac{1}{Z}\psi_{AB}(x_A, x_B)\psi_{BC}(x_B, x_C)}{\sum_{x'_{A}}\sum_{x'_{C}}\frac{1}{Z}\psi_{AB}(x'_{A}, x_B)\psi_{BC}(x_B, x'_{C})}\\
&=\frac{\psi_{AB}(x_A, x_B)}{\sum_{x'_{A}}\psi_{AB}(x'_{A}, x_B)}\frac{\psi_{BC}(x_B, x_C)}{\sum_{x'_{C}}\psi_{BC}(x_B, x'_{C}) }
\end{split}
\end{equation*}

\begin{equation*}
\begin{split}
p(A|B)&=\frac{p(A,B)}{p(B)}=\frac{\sum_{x'_C}p(x_A, x_B, x'_C)}{\sum_{x'_A}\sum_{x'_C}p(x'_A, x_B,x'_C)}\\
&=\frac{\sum_{x'_C}\frac{1}{Z}\psi_{AB}(x_A, x_B)\psi_{BC}(x_B, x'_C)}{\sum_{x'_A}\sum_{x'_C}\frac{1}{Z}\psi_{AB}(x'_A, x_B)\psi_{BC}(x_B, x'_C)}\\
&=\frac{\psi_{AB}(x_A, x_B)\sum_{x'_C}\psi_{BC}(x_B, x'_C)}{\sum_{x'_A}\psi_{AB}(x'_A, x_B)\sum_{x'_C}\psi_{BC}(x_B, x'_C)}\\
&=\frac{\psi_{AB}(x_A, x_B)}{\sum_{x'_A}\psi_{AB}(x'_A, x_B)}
\end{split}	
\end{equation*}
同理可得:
$p(C|B)=\frac{\psi_{BC}(x_B, x_C)}{\sum_{x'_C}\psi_{BC}(x_B, x'_C)}$

故有$p(A,C|B)=p(A|B)p(C|B)$
\end{itemize}

\end{document}
